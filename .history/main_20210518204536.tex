% !Mode:: "TeX:UTF-8"
\documentclass[type=master, openany, pifootnote]{shuthesis}
% 选项:
%  type=[master|doctor],            % 必选
%  secret,                          % 可选 (如果论文需要保密, 这一项需要打开)
%  pifootnote,                      % 可选 (建议打开)
%  openany|openright,               % 可选 (章首页是右开还是任意开, 默认是右开)
%  nocolor                          % 提交最终版本时请打开此选项
\usepackage{shuthesis}
\usepackage{adjustbox}
\usepackage{graphicx}
\graphicspath{{figures/}}
\linespread{1.2}

\begin{document}
\newcommand{\upcite}[1]{\textsuperscript{\cite{#1}}}

\frontmatter
\shusetup{
    %%%%%%%% 1. 配置里面不要出现空行. 2. 不需要的配置信息可以删除. %%%%%%%%%%%
    secretlevel={公开},             % 秘级
    ctitle={H.265 无损帧内编码算法优化及硬件实现},
    etitle={Algorithm Optimization and Hardware Implementation of H.265 Lossless Intra Frame Coding},
    cdisciplines={工学},
    edisciplines={Engineering},
    catalognumber={TN919.81},
    cdate={2021 年 5 月},
    edate={May, 2021},
    coverdate={\zhdigits{2021}年\zhnumber{5}月},
    cdepartment={通信与信息工程学院},
    edepartment={School of Communication and Information Engineering},
    cmajor={信号与信息处理},
    emajor={Signal and information processing},
    % 盲审隐藏
    id={18721542},
    cauthor={林庆毫},
    eauthor={Qinghao Lin},
    csupervisor={林敏},
    esupervisor={Min Lin},
    % id={xxxxxxxx},
    % cauthor={xxxxxx},
    % eauthor={xxxxxx},
    % csupervisor={xxxxxx},
    % esupervisor={xxxxxx},
}

\input{data/abstract}
% \makefirstpage                         % 生成带有学校 logo 的封面  有编译问题 全部完成后取消注释 手动跳过编译问题 (Linux xelatex 可过)
\makecover

\tableofcontents                       % 目录
% \input{data/denotation}              % 符号对照表

% 正文部分
\mainmatter
% !Mode:: "TeX:UTF-8"
\chapter{绪论}
\label{cha:c1}

\section{课题研究背景与意义}

\section{国内外研究现状}

\section{本课题的主要工作}

\section{本文章节安排}
测试引用\cite{tex1989}。
% !Mode:: "TeX:UTF-8"
\chapter{HEVC 视频压缩原理与关键技术}
\label{cha:c2}

\section{HEVC 视频压缩技术发展历史}

\section{HEVC 帧内压缩技术}

\section{HEVC 帧间压缩技术}

\section{HEVC 软硬件实现开源项目}
% !Mode:: "TeX:UTF-8"
\chapter{HEVC 帧内无损压缩优化算法}
\label{cha:c3}

\section{HEVC 帧内预测过程分析}

\section{HEVC 帧内无损压缩的预测过程优化}

\section{HEVC 帧内分块决策过程分析}

\section{HEVC 帧内无损压缩的分块过程优化}

\section{HEVC 帧内无损压缩优化算法性能测试}
% !Mode:: "TeX:UTF-8"
\chapter{算法的硬件实现}
\label{cha:c4}

\section{视频编码硬件实现的必要性}

\section{各模块的硬件实现}

\section{硬件实现仿真与分析}
\include{data/chap05}
\include{data/chap06}
\listoffigures                        % 插图索引
\listoftables                         % 表格索引
% 参考文献
\bibliographystyle{shuthesis}
\bibliography{reference/大论文写作}

% 注意盲审隐藏、修改
\include{data/publications}           % 发表论文清单
% 致谢
\begin{acknowledgement}
    衷心感谢导师xxx教授对本人的精心指导。
\end{acknowledgement}
        % 致谢

\backmatter
% \begin{appendix}
%     \input{data/appendix}
% \end{appendix}

\end{document}


% 手动处理内容备忘
% 盲审记得隐藏发表论文清单
% 盲审还要记得把致谢处理了
% 打开 makefirstpage