% !Mode:: "TeX:UTF-8"
\chapter{H.265 帧内无损编码 FPGA 原型验证平台}
\label{cha:c5}
当电路系统复杂度高于一定程度时,传统的行为级仿真已经难以满足设计的验证需求,使用 FPGA 进行原型验证灵活度高,验证结果可靠,且有利于软硬件协同设计、加快项目开发进度。因此 FPGA 原型验证成为了复杂系统 ASIC 化之前不可或缺的一个步骤。

\section{FPGA 原型验证方案}
为了对设计的硬件编码器进行验证,本课题设计、搭建了一个完整的 FPGA 原型验证平台。该验证平台由视频源、预处理模块、通信模块、硬件编码器和上位机解码模块组成,利用该平台可实时编码由视频源提供的图像信息,经映射入 FPGA 的硬件编码器编码后将码流传回上位机,在上位机对码流进行网络适配层码流的拼接后,由上位机软解码,实时查看软解码结果正确与否即可对硬件编码器进行验证。完整的验证平台如图 \ref{fig:FPGADemoBlock} 所示。

\section{验证平台关键模块}
\subsection{FPGA 平台}
\% 简要介绍 FPGA 开发板的可用资源,说明 LUT RAM DSP 等足够映射编码器 \%

\subsection{视频源及预处理}

% 文件 摄像头 显卡HDMI输出
\subsection{与上位机通信接口}
% USB串口CP2102 以太网口 PCIE

\subsection{编码模块}
% ultra ram DSP blockram distributionram FIFO

\subsection{上位机解码模块}
% VCL NAL

\section{FPGA 原型验证结果}
% 照片
% 说明摄影环境

\section{本章小结}