% !Mode:: "TeX:UTF-8"
\chapter{视频编码原理与关键技术}
\label{cha:c2}
% 编码 与 压缩 用词考虑

\section{视频编码技术发展历史}
% 自媒体 短视频 AR/VR 疫情下的远程会议 远程KTV Steam的远程同玩

\section{预测编码技术}
\subsection{帧内编码}
\subsection{帧间编码}

\section{变换与量化技术}
\subsection{变换编码}
\subsection{量化}

\section{环路后处理技术}
\subsection{去方块滤波}
\subsection{样点自适应补偿}

\section{熵编码技术}
\subsection{变长编码在视频编码中的应用}
\subsection{算术编码在视频编码中的应用}
\subsection{模式依赖的编码顺序}

\section{率-失真优化技术}
% 标准只规定了码流语义 编码可以自由发挥 就算我全用64x64质量超低也能叫HEVC 所以需要优秀的压缩率-失真的折衷策略

\section{视频编码软硬件实现开源项目}