% !Mode:: "TeX:UTF-8"
\chapter{绪论}
\label{cha:c1}

\section{课题研究背景与意义}

\section{国内外研究现状}

\section{本课题的主要工作}

\section{本文章节安排}
本文由 6 个章节组成。
本章概述课题研究背景与意义,介绍本课题相关的国内外研究状况,并简要阐述本课题的工作。
第 2 章首先从整体上介绍视频编码技术的发展历史,其中主要介绍 H.265/HEVC,亦会涉及到其前身 H.264/AVC 及近年仍在草拟标准的 H.266/VVC;同时从帧内/帧间预测、变换量化、熵编码、环路滤波和率-失真优化 5 个视频编码的核心手段展开,简单介绍技术要点;最后介绍学术界和工业界常用的软硬件视频编解码相关的开源项目,本课题就是以其中部分开源项目为基础进行的。
第 3 章是本文的核心部分之一,首先详细地分析了帧内无损压缩优化的可行性,通过分析提出了 3 个可优化的方向,

以往针对大型零部件的虚拟加工,因文件量的巨大其系统的构建大都以 SGI 等图形工作站做为硬件平台。SGI 等图形工作站虽然可以达到非常理想的视觉效果,但是它的成本也同样是高昂的。对于此,在不影响虚拟加工的效果情况下,本文所构建的硬件平台是建立在普通 PC 机基础上,其构成如图 3.1 所示。(注:图置中)