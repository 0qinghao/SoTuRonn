% !Mode:: "TeX:UTF-8"
\chapter{绪论}
\label{cha:c1}

\section{课题研究背景与意义}
% 自媒体 短视频 AR/VR 疫情下的远程会议 在线多人KTV Steam的远程同玩 远程看房

\section{国内外研究现状}


\section{本课题的主要工作}
本课题的主要目的是提高 H.265 帧内编码效率,即在保证视频图像质量不变的情况下,提高编码后的压缩率。为此,本文对 H.265 的核心编码过程进行了详尽的分析,提炼出 3 个可优化的方向并进行了软、硬件实现,通过测试验证达到了预期的性能指标。本课题的主要工作内容概述如下:
\begin{enumerate}
    \item 整体上介绍视频编码的基本原理,解释帧内/帧间预测、变换量化、熵编码、环路滤波和率-失真优化 5 个核心手段在视频编码过程中发挥的作用;
    \item 进一步分析帧内预测过程及原理,针对其部分待预测像素距离参考像素距离过远的缺陷,提出了一种 L 形迭代预测 (L-shape-based Iterative Prediction, LIP) 算法,将待预测像素与参考像素的距离缩短到 1,极大地提高了帧内预测的准确性,从而提高编码效率;
    \item 进一步分析帧内分块决策过程,配合上述 LIP 算法,提出了一种不同于 H.265 传统的基于四叉树结构的分块方法,即 L 形分块 (L-shape-based Block Partitioning, LBP) 算法。LIP 与 LBP 互相配合,进一步提高编码效率;
    \item 进一步分析系数的编码过程,利用待编码系数中存在的特殊的空间结构性,提出了一种残差中值边缘检测 (Residuals Median Edge Detect, RMED) 算法,通过对预测残差进行二次处理降低待编码系数的整体能量,进一步提高编码效率;
    \item 在开源 H.265 硬件编码器的基础上,实现上述改进算法,同时利用现场可编程逻辑门阵列 (Field Programmable Gate Array, FPGA) 搭建完整的硬件编、解码系统原型验证平台,对提出的改进算法进行测试验证。
\end{enumerate}

\section{本文章节安排}
本文由 6 个章节组成,具体安排如下:

本章为绪论,概述课题研究背景与意义,介绍本课题相关的国内外研究状况,并简要阐述本课题的工作。

第 2 章首先从整体上介绍视频编码技术的发展历史,其中主要介绍 H.265,亦会涉及到其前身 H.264 及近年仍在草拟标准的 H.266;同时从帧内/帧间预测、变换量化、熵编码、环路滤波和率-失真优化 5 个视频编码的核心手段展开,简单介绍技术要点;最后介绍部分学术界和工业界常用的软、硬件视频编解码相关的开源项目,本课题就是以其中部分开源项目为基础开展的。

第 3 章是本文的核心部分之一,该章详细地探讨了帧内无损编码优化的可行性。通过对帧内预测过程、帧内分块决策过程和系数编码过程的进一步分析,提出了 3 个优化算法,分别涉及 1) 帧内预测过程的优化;2) 帧内分块过程的优化;3) 待编码系数的再处理。最后对 3 种优化算法进行单独、联合测试,给出优化算法的性能测试结果。

第 4 章开始介绍硬件实现,详细描述了所提出的优化算法在开源 H.265 硬件编码器上的实现框架和具体模块的实现方案,并给出了行为级仿真结果。

第 5 章介绍完整的 H.265 硬件编、解码系统的 FPGA 原型验证平台,与算法的软件实行进行对比,对验证结果进行分析。

第 6 章对本课题的全部工作进行总结,并针对下一步的研究工作作出展望。
