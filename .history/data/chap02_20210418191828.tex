% !Mode:: "TeX:UTF-8"
\chapter{视频编码原理与关键技术}
\label{cha:c2}
在正文之前,首先申明以下用词:1) 统一使用视频“编码”而非视频“压缩”。在视频编解码领域,“编码”与“压缩”表达相同的含义,都指经过预测、变换量化、熵编码等操作达到压缩视频数据量的目的。编码是手段,压缩是目的,按照习惯统一使用视频编码这一表述,类似地,使用编码标准、编码方案、编码效率等表述;2) 
% 编码 与 压缩 用词考虑
% 265 HEVC 统一 265
% VTM 不用 JEM

\section{视频编码技术发展历史}
% 自媒体 短视频 AR/VR 疫情下的远程会议 在线多人KTV Steam的远程同玩 远程看房

\section{预测编码技术}
\subsection{帧内编码}
\subsection{帧间编码}

\section{变换与量化技术}
\subsection{变换编码}
\subsection{量化}

\section{环路后处理技术}
\subsection{去方块滤波}
\subsection{样点自适应补偿}

\section{熵编码技术}
\subsection{变长编码在视频编码中的应用}
\subsection{算术编码在视频编码中的应用}
\subsection{模式依赖的编码顺序}

\section{率-失真优化技术}
% 标准只规定了码流语义 编码可以自由发挥 就算我全用64x64质量超低也能叫HEVC 所以需要优秀的压缩率-失真的折衷策略

\section{软、硬件视频编码开源项目}