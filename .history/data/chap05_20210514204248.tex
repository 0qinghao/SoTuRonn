% !Mode:: "TeX:UTF-8"
\chapter{总结与展望}
\label{cha:c5}

\section{课题总结}

\section{工作展望}
% 目前预测残差再处理模块是紧紧放在熵编码前的,可以考虑把模块放到预测后,使其加入 PU 级别的率-失真优化过程
% 类似地 可以考虑帧间预测残差的应用
针对 H.265 帧内编码效率的优化,本课题还有多项想要尝试的实验与优化方案,记录如下。期望本课题在同窗的后续耕耘下能够百尺竿头,更进一步。
\begin{enumerate}
    \item 碍于迭代预测时对重建点的需求,L-IP 算法目前仅能应用在无损编码中。将其应用在有损编码的一个方案是寻找一种合适的一维变换,使能够以 L 形区域的结构完成变换-量化-去量化-逆变换回路。如何找到一种去相关性能良好的一维变换是亟待解决的问题;
    \item R-MED 算法经过简单测试发现其在大分辨率图像或屏幕图像的有损编码中能得到 2\%\textasciitilde 3\% 的编码效率优化。有损编码中待编码系数已不具有常规的空间相关性,此时 R-MED 仍能改进编码效率,其中机理需要进一步分析梳理,经过进一步改进后有望将其应用到有损编码中。类似地,在帧间编码中运动补偿残差又有其独特的数据相关性,能否在其上应用 R-MED 值得做实验尝试;
    \item 本课题提出的算法理论上均可应用在最新的 H.266 标准中。H.266 在帧内预测和编码块划分中加入了大量新机制,如何将所提算法与新机制有机结合,使其在新标准中得到更高的编码效率优化是一个值得研究的内容。
\end{enumerate}