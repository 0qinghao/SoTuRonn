% !Mode:: "TeX:UTF-8"
\chapter{H.265 帧内无损编码及优化算法硬件实现}
\label{cha:c4}
在视频实时编码的场景中,使用专用集成电路 (Application Specific Integrated Circuit, ASIC) 实现可做到功耗与性能的良好平衡,是业界的主流做法。本章介绍 H.265 帧内无损编码的 ASIC 实现,并给出行为级仿真结果。

\section{硬件系统框架}
如图 \ref{fig:HardwareArch} 所示,H.265 帧内无损编码器包含的主要模块有:帧内预测模块(分为 2 个阶段进行)、熵编码模块、环路后处理模块、变换量化模块(无损编码时跳过处理)以及与外部存储器交互的 Fetch 与像素 Buffer 模块,同时存在大量用于待编码系数的中间缓存模块,不一一列出。
\begin{figure}[hbt]
    \centering
    \includegraphics{HardwareArch.pdf}
    \caption{H.265 编码器硬件系统框架}
    \label{fig:HardwareArch}
\end{figure}

图中的外部 CPU 并非系统框架内的模块,系统框架预留了与外部控制器的交互接口,用于进行诸如感兴趣区域 (Range of Interest, ROI)、码率控制、时延控制等外部控制操作。

\section{关键模块的硬件实现}
\begin{itemize}
    \item 帧内预测模块

          根据前文所述,H.265 标准下在进行帧内预测时,需要对一个 PU 内的 35 种预测模式进行扫描,从而判断哪种模式是最优的。然而在硬件实现中出现一个问题:在进行预测之前,需要获取其经过重建的参考像素,而参考像素大概率位于前一个 PU 之中,因此,为了等待重建需要让整个预测处理在时序中长时间停留,导致系统吞吐率和硬件的利用率降低,如图 \ref{fig:Intra12}(a) 所示。这一情况在处理大尺寸预测单元时尤为明显。
          \begin{figure}[hbt]
              \centering
              \includegraphics{Intra12.pdf}
              \caption{帧内预测时序说明}
              \label{fig:Intra12}
          \end{figure}
          基于上述分析,使用了如图 \ref{fig:Intra12}(b) 所示的


\end{itemize}

\section{行为级仿真与测试}
% 软件编码结果文件比对
% TSMC 65 nm 400MHz 71+138+764+198+125+103K Logic gate count

\section{本章小结}