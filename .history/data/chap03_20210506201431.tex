% !Mode:: "TeX:UTF-8"
\chapter{帧内无损编码优化算法}
\label{cha:c3}
本章从帧内无损编码的可优化方向的分析开始,引出本研究课题中提出的 3 个优化算法:。在具体描述所提出算法之前均先详细地分析了标准规定或 HM 参考软件中的实现方案,以与所提算法形成对比。最后在对各算法性能单独测试的基础上,给出量化算法的性能测试,以证明所提算法的有效性。

\section{帧内无损编码的可优化方向}

\section{帧内预测过程分析}
\label{cha:IntraPredDetail}

\begin{figure}[hbt]
    \centering
    \includegraphics{IntraAngModeOverview.pdf}
    \caption{H.265 帧内预测角度模式}
    \label{fig:IntraAngModeOverview}
\end{figure}

\begin{figure}[hbt]
    \centering
    \includegraphics{IntraProjection.png}
    \caption{H.265 帧内预测结果示例}
    \label{fig:IntraProjection}
\end{figure}
% 可能会滤波 所以不是完全投影

\section{帧内无损编码的预测过程优化}

\section{帧内分块决策过程分析}

\section{帧内无损编码的分块过程优化}

\section{系数编码过程分析}

\section{待编码系数的再处理}

\section{联合算法性能测试}