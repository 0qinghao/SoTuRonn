% !Mode:: "TeX:UTF-8"
\chapter{视频编码原理与关键技术}
\label{cha:c2}
在正文之前,首先申明以下用词:1) 统一使用视频“编码”而非视频“压缩”。在视频编解码领域,“编码”与“压缩”表达相同的含义,都指经过预测、变换量化、熵编码等操作达到减小数据冗余、压缩视频数据量的目的。编码是手段,压缩是目的,按照习惯统一使用视频编码这一表述,类似地,使用编码标准、编码方案、编码效率等表述;2) 为了直接体现出编码标准的发展历程,不使用高级视频编码(Advanced Video Coding,AVC)、高性能视频编码(High Efficiency Video Coding,HEVC)和多功能视频编码(Versatile Video Coding,VVC)的表述,统一使用 H.264、H.265 和 H.266,两类表述在含义上并无差异;3) 由联合专家组维护的各标准参考模型随着版本迭代可能存在不同的命名,例如 H.266 的参考模型在早期曾称为 JEM(Joint Exploration Model),本文中统一使用以下命名:用于 H.264 的 JM(Joint Test Model)模型、用于 H.265 的 HM(HEVC Test Model)模型以及用于 H.266 的 VTM(VVC Test Model)模型。

\section{视频编码技术发展历史}

\section{预测编码技术}
\subsection{帧内编码}
\subsection{帧间编码}

\section{变换与量化技术}
\subsection{变换编码}
\subsection{量化}

\section{环路后处理技术}
\subsection{去方块滤波}
\subsection{样点自适应补偿}

\section{熵编码技术}
\subsection{变长编码在视频编码中的应用}
\subsection{算术编码在视频编码中的应用}
\subsection{模式依赖的编码顺序}

\section{率-失真优化技术}
不论是视频编码标准还是各类网络传输协议标准,标准中制定的往往只是各类语法和语义(部分包含同步规则和检错纠错规则),只要是符合标准语法语义的码流都能称为符合某视频编码标准。例如在使用 H.265 进行编码时,若将最小预测单元限定为 32x32(HM 默认配置是 4x4),编码效率与失真度可以预见甚至不如简单的 M-JPEG(Motion Joint Photographic Experts Group)。因此,如何在标准的编码框架下得到最优的编码效率与失真度的平衡是编码算法研究的核心内容之一,寻找折衷点的过程被称为率-失真优化(Rate-Distortion Optimization,RDO)。
% 有各种级别的率-失真优化,片层级,CTU 级,PU 级
% 不同的软件有不同的方案,介绍 HM
\begin{figure}[htb]
    \centering
    \includegraphics[scale=1.0]{Qtree.pdf}
    \caption{四叉树}
\end{figure}
% 标准只规定了码流语义 编码可以自由发挥 就算我全用64x64质量超低也能叫HEVC 所以需要优秀的压缩率-失真的折衷策略
% 也是业界在竭力优化的核心内容

\section{软、硬件视频编码开源项目}
互联网开源精神是当今学术界与工业界发展的重要推手,而视频编解码这一涉及到成像系统、生物信息、数学、统计学、计算机以及集成电路的庞大学科,如果没有一个系统的开源项目让各学科的研究人员参与进来,而是由单一的某一组织闭门造车是无法得到长足发展的。本小节总结了目前常用的软、硬件视频编码开源项目,本课题就是基于其中部分项目开展的。
\subsection{软件视频编码开源项目}
\begin{itemize}
    \item JM

    JM 是联合开发组(Joint Video Team,JVT)负责维护的 H.264 编解码参考软件\citeup{AVCsoftwareJM}。

    \item x264

    x264 是

    \item HM
\end{itemize}

\subsection{硬件视频编码开源项目}