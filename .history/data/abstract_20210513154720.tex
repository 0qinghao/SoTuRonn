% 中英文摘要和关键字
\begin{cabstract}

    本课题通过对 H.265 无损帧内编码的深入研究,归纳出其在帧内预测和编码块划分流程中存在的缺陷,针对性地提出优化算法。课题进行的工作简述如下:

    软件算法方面。1) 针对帧内预测在远离参考点的区域预测准确度低的缺陷,提出 L 形迭代预测算法,使所有待预测点与参考点的距离缩短到 1 单位,大幅提高帧内预测准确度;2) 针对四叉树结构下的编码块划分无法适应小片区域复杂纹理的缺陷,提出 L 形分块算法,使编码块划分更加灵活,同时与 L 形迭代预测有机融合,提出联合算法,进一步提高编码效率;3) 针对帧内预测残差仍具有特殊空间冗余的缺陷,提出残差中值边缘检测算法,对残差进行二次预测,有效降低其空间冗余,提高编码效率。

    硬件实现方面。
\end{cabstract}

\ckeywords{无损视频编码, H.265, 帧内预测, 编码块划分, 硬件实现}

\begin{eabstract}
    Abstract in English.
\end{eabstract}

\ekeywords{\TeX, \LaTeX, Template, Thesis}
