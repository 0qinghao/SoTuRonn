% 致谢
\begin{acknowledgement}
    % 做研究不要浮躁 要冷静下来钻到经典文献中去 大开眼界 多了解各领域的研究内容、发展状况
    % 作为实验室第一批做多媒体的成员 深感自己能力不足
    衷心感谢恩师林敏教授对本人的精心指导。恩师阅历之深,涉猎之广,经验之丰富,思考问题之周全无不让我钦佩。
    仍记得初入研究室时,恩师便教导我们做科研不要浮躁,不要只追求一些“快知识”,要冷静下来钻到经典的、高水平的书籍和文章中去,要多了解身边同窗的研究内容、行业的发展状况。恩师的教导让我从只会看博客、查百科转变成了会看文献、会思考问题的合格的研究生。若无恩师引导鞭策,三载光阴必将落得碌碌无为。

    衷心祝愿研究室同窗百尺竿头更进一步。作为研究室首届成员之一,求学之路难免有筚路蓝缕之艰辛,正是有你们的陪伴和帮助,才让我有了直面困难的勇气,打消了偷懒、放弃的念头。还要特别感谢图像组的两位同窗,正是有你们的协助我才能顺利完成此文,相信你们能带领图像组创造佳绩。

    对导师、同窗致谢的同时还应对自己进行反思。作为实验室 想学的太多,精通的太少。疫情期间自我约束能力差。缺乏大局观。

    囿于篇幅,不再一一言谢,惟愿大家科研顺利、工作顺利、身体安康。
\end{acknowledgement}
