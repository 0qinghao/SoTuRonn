% !Mode:: "TeX:UTF-8"
\chapter{总结与展望}
\label{cha:c6}

\section{课题总结}
本课题的目的是提高 H.265 无损帧内编码效率。为此,本课题对 H.265 的帧内预测过程、编码块划分过程进行了详尽的分析,提出 4 个优化算法并进行了软、硬件实现。测试验证结果显示达到了预期的性能指标。本课题的主要工作内容总结如下:
\begin{enumerate}
    \item 分析帧内预测过程及原理,针对其部分待预测像素距离参考点距离过远的缺陷,提出了一种 L 形迭代预测算法,将待预测像素与参考点的距离缩短到 1,极大地提高了帧内预测的准确性,从而提高编码效率;
    \item 分析编码块划分过程及原理,提出了一种不同于 H.265 基于四叉树结构的分块方法,即 L 形分块算法。同时提出了将其与 L 形迭代预测算法融合的联合算法,进一步提高编码效率;
    \item 分析系数的编码过程,利用待编码系数中存在的特殊的空间结构性,提出了一种残差中值边缘检测算法,通过对预测残差进行二次处理降低待编码系数的整体能量,进一步提高编码效率;
    \item 在 H.265 硬件编码器的基础上,实现上述改进算法,同时利用 FPGA 搭建硬件编码系统原型验证平台,使用该平台可对提出的改进算法进行测试验证。
\end{enumerate}

\section{工作展望}
% 目前预测残差再处理模块是紧紧放在熵编码前的,可以考虑把模块放到预测后,使其加入 PU 级别的率-失真优化过程
针对 H.265 帧内编码效率的优化,本课题还有多项想要尝试的实验与优化方案,记录如下。期望本课题在同窗的后续耕耘下能够百尺竿头,更进一步。
\begin{enumerate}
    \item 碍于迭代预测时对重建点的需求,L-IP 算法目前仅能应用在无损编码中。将其应用在有损编码的一个方案是寻找一种合适的一维变换,使能够以 L 形的结构完成变换-量化-去量化-逆变换回路。如何找到一种去相关性能良好的一维变换是亟待解决的问题;
    \item R-MED 算法经过简单测试发现其在大分辨率图像或屏幕图像的有损编码中能得到 2\%\textasciitilde 3\% 的编码效率优化。有损编码中待编码系数已不具有常规的空间相关性,此时 R-MED 仍能改进编码效率,其中机理需要进一步分析梳理,经过后续改进后有望将其应用到有损编码中。类似地,在帧间编码中运动补偿残差又有其独特的数据相关性,能否在其上应用 R-MED 值得做实验尝试;
    \item 本课题提出的算法理论上均可应用在最新的 H.266 标准中。H.266 在帧内预测和编码块划分中加入了大量新机制,如何将所提算法与新机制有机结合,使其在新标准中得到更高的编码效率优化是一个值得研究的内容。
\end{enumerate}