% 中英文摘要和关键字
\begin{cabstract}
    H.265 是联合专家组制定的视频编码标准,相比 H.264 提高了 50\% 的编码效率,已广泛应用在安防、视频会议、网络视频发布等产业中。帧内编码是 H.265 的重要组成部分,其编码效率直接影响到关键帧的压缩率。    
    本课题通过对 H.265 无损帧内编码的深入研究,归纳出其在帧内预测和编码块划分流程中存在的缺陷,针对性地提出优化算法。课题进行的工作简述如下:

    软件算法方面。1) 针对帧内预测在远离参考点的区域预测准确度低的缺陷,提出 L 形迭代预测算法,使所有待预测点与参考点的距离缩短到 1 单位,大幅提高帧内预测准确度;2) 针对四叉树结构下的编码块划分无法适应小片区域复杂纹理的缺陷,提出 L 形分块算法,使编码块划分更加灵活,同时与 L 形迭代预测有机融合,提出联合算法,进一步提高编码效率;3) 针对帧内预测残差仍具有特殊空间冗余的缺陷,提出残差中值边缘检测算法,对残差进行二次预测,有效降低其空间冗余,提高编码效率。

    硬件实现方面。在开源 H.265 硬件编码器的基础上,实现上述改进算法,同时利用 FPGA 搭建硬件编码系统原型验证平台,利用该平台可对提出的各类改进算法进行硬件级测试验证。

    基于联合专家组制定的通用测试环境和测试序列对本课题所提算法进行测试。实验结果表明,与 H.265 标准测试模型相比,上述 3 个算法分别得到了 9.31\%, 2.34\%, 7.04\% 的码率优化,联合算法得到了 10.21\% 的码率优化,证明所提算法性能良好。在 FPGA 验证中,得到了与软件测试相当的优化效果,证明所提算法硬件实现可行,有较高的应用价值。
\end{cabstract}

\ckeywords{无损视频编码, H.265, 帧内预测, 编码块划分, 硬件实现}

\begin{eabstract}
    Abstract in English.
\end{eabstract}

\ekeywords{\TeX, \LaTeX, Template, Thesis}
