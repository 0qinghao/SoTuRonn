% !Mode:: "TeX:UTF-8"

\chapter{绪论}
\label{cha:c1}

\section{课题研究背景与意义}
% 自媒体 短视频 AR/VR 疫情下的远程会议 在线多人KTV Steam的远程同玩 远程看房 NVR/DVR
H.264/AVC(Advanced Video Coding)、H.265/HEVC(High Efficiency Video Coding)、H.266/VVC(Versatile Video Coding) 是由国际电信联盟与国际标准化组织联合专家组制定的一系列视频压缩编码标准\upcite{H266Overview,H265Overview,H264Overview},是数字视频编码技术 40 年的学术研究和 30 年的标准化的成果,代表着自 2003 年至今各时期最先进的视频编码技术。
帧内编码是 H.26X 系列标准中的重要组成部分,其利用图像的空间相关性进行数据压缩,视频的关键帧也只使用帧内编码处理。
加之近年来,在传感器分辨率不断提高、存储价格不断下降的催化下,自动驾驶视觉、云游戏、后期制作、影像存档等需要使用无损编码的应用加速发展\upcite{LatestLosslessIntraCodingAsRef}。
同时,在某些高精度需求的场景中无损编码仍是不可或缺的存在,例如来之不易的卫星遥感图像的编码、涉及安全保障的指纹图像存储、全球疫情影响下的医学影像传输、远程桌面共享等。
因而探索无损帧内编码的优化算法有很高的研究和应用价值。

本课题的另一项重点研究内容是 H.265 编码器及优化算法的硬件实现。实时视频编码在监控、直播等场景中应用广泛,实时编码一般有 2 类实现方案:1) 利用数字信号处理器 (Digital Signal Processor, DSP) 实现;2) 使用专用视频编码硬件实现,例如各类手机处理器中集成的视频编解码单元。由于 H.265 实时编码对硬件的性能和功耗需求都极高,使用 DSP 实现难度极大,实现后编码能力低。专用视频编码器经优化后可得到良好的性能与功耗的折衷,因此研究专用 H.265 编码器的硬件实现有很高的应用价值。

\section{国内外研究现状}
H.26X 系列编码标准可以通过简单地跳过变换、量化、去块滤波、自适应样点补偿等可能引入失真的步骤,实现无损帧内编码\upcite{BypassImprovingSCC}。但也因此使得待编码系数具有较高的能量,为后续的熵编码带来极大的压力。为优化H.26X系列标准无损帧内编码的性能,国内外学者进行了大量的研究。
Kamisli\upcite{LosslessI2ITransformTCSVT} 提出了一种不会引入失真的整数离散正弦变换,可在不增加数据动态范围的情况下将样点值映射到变换域,并且可用于无损编码,一定程度上弥补了帧内无损编码时跳过变换带来的码率损失;
Zhang 等\upcite{CrossComponentPredictionCCLM}分析了亮度通道与色差通道之间的关系,得出编码块中亮度和色差分量之间存在线性相关性的结论,据此设计了3个模型,使用亮度通道对色差通道进行帧内预测,一定程度上降低了需要编码的数据量;
残差差分脉冲编码调制 (Residual Differential Pulse Code Modulation, RDPCM) 是 H.265 屏幕图像编码扩展标准 (HEVC Screen Content Coding Extension, HEVC-SCC)\upcite{HEVCSCCOverview} 的一部分,通过对残差进行再处理的方式进一步去除残差间的相关性,从而提高了编码效率\upcite{docpaperSCC};
区块差分脉冲编码调制 (Block Differential Pulse Code Modulation, BDPCM)\upcite{H266Overview} 是 H.266 标准的一部分,通过在水平或垂直方向上使用临近像素迭代预测的方式提高了预测准确性;
Zhou 等\upcite{SAP-SAPE,SAP-SAP1,SAP-SAPHVSWP2DTM,SAP-SAPHV,SAP-SAP}将 DPCM 的思想扩展到了任意预测角度,提出了基于样本点的帧内角度预测 (Sample-based Angular Intra-Prediction, SAP) 及其多种衍生方案,使用逐点进行预测-重建的方案代替整块处理的原始方案,进一步提高了预测准确性;
Sanchez 等\upcite{pwmResidualsPiecewiseMapping}通过分析残差块中数据的分布特征,提出了将待编码数据的“重心”进行搬移的新颖方案,其针对不同的分布特征设计了 3 组映射规则,整体降低了残差的能量;
元辉等\upcite{XiDianIntraPredictionH264}提出若在编码时发现当前预测单元各预测模式的结果高度相似,则不必编码模式,而是将所有预测模式的均值作为预测结果,一定程度上减少了需要编码的模式信息;
Li 等\upcite{EfficientMultiplelinebasedIntra}发现使用最靠近预测单元的单行单列样点作为参考点的方式有时无法得到最佳预测结果,反而部分远距离参考点更有利于进行预测,因而提出通过将参考点拓展到多行的方式提高帧内预测的准确性。

上述研究中提出的算法对帧内编码的性能均有不同程度的优化,但仍存在以下待改进的问题:
\cite{LosslessI2ITransformTCSVT} 设计的整数变换方法局限性很大,且没有量化的配合很难降低待编码系数的能量,因此带来的码率优化较小;
\cite{pwmResidualsPiecewiseMapping,HEVCSCCOverview} 设计的残差分段映射方案没有考虑残差独特的空间相关性,仍有优化空间;
\cite{CrossComponentPredictionCCLM,SAP-SAPE,SAP-SAP1,SAP-SAPHVSWP2DTM,SAP-SAPHV,SAP-SAP,XiDianIntraPredictionH264,EfficientMultiplelinebasedIntra} 均是设法增加或优化预测模式来提高帧内预测的准确性,然而各代 H.26X 标准中的帧内预测模式不断增多,在最新的 H.266 中已经达到 67 种\upcite{VVCComplexityAnalysisEncodeTime30xDecodeTime3x},很难再挖掘出优化空间,可以预见今后这类改进带来的优化会越来越小;
最后,上述大部分方案具有一定的应用范围的局限性,很难在各世代的 H.26X 标准中通用。

\section{本课题的主要工作}
本课题的主要目的是提高 H.265 无损帧内编码效率,即在保证视频经过编解码后无失真的情况下,提高编码后的压缩率。为此,本文对 H.265 的核心编码过程进行了详尽的分析,提炼出 3 个可优化的方向并进行了软、硬件实现,通过测试验证达到了预期的性能指标。本课题的主要工作内容概述如下:
\begin{enumerate}
    \item 整体上介绍视频编码的基本原理,解释帧内/帧间预测、变换量化、熵编码、环路滤波和率-失真优化 5 个核心手段在视频编码过程中发挥的作用;
    \item 进一步分析帧内预测过程及原理,针对其部分待预测像素距离参考点距离过远的缺陷,提出了一种 L 形迭代预测 (L-shape-based Iterative Prediction, LIP) 算法,将待预测像素与参考点的距离缩短到 1,极大地提高了帧内预测的准确性,从而提高编码效率;
    \item 进一步分析帧内分块决策过程,配合上述 LIP 算法,提出了一种不同于 H.265 传统的基于四叉树结构的分块方法,即 L 形分块 (L-shape-based Block Partitioning, LBP) 算法。LIP 与 LBP 互相配合,进一步提高编码效率;
    \item 进一步分析系数的编码过程,利用待编码系数中存在的特殊的空间结构性,提出了一种残差中值边缘检测 (Residuals Median Edge Detect, RMED) 算法,通过对预测残差进行二次处理降低待编码系数的整体能量,进一步提高编码效率;
    \item 在 H.265 硬件编码器的基础上,实现上述改进算法,同时利用 FPGA(Field Programmable Gate Array) 搭建硬件编码系统原型验证平台,对提出的改进算法进行测试验证。
\end{enumerate}

\section{本文章节安排}
本文由 6 个章节组成,具体安排如下:

本章为绪论,概述课题研究背景与意义,介绍本课题相关的国内外研究状况,并简要阐述本课题的工作。

第 2 章首先从整体上介绍视频编码技术的发展历史,其中主要介绍 H.265,亦会涉及到其前身 H.264 及 2020 年正式发布的 H.266;同时从帧内/帧间预测、变换量化、熵编码、环路滤波和率-失真优化 5 个视频编码的核心手段展开,简单介绍技术要点;最后介绍部分学术界和工业界常用的软、硬件视频编解码相关项目,本课题就是以其中部分项目为基础开展的。

第 3 章是本文的核心部分之一,该章详细地探讨了帧内无损编码优化的可行性。通过对帧内预测过程、帧内分块决策过程和系数编码过程的进一步分析,提出了 4 个优化算法,分别涉及 1) 帧内预测过程的优化;2) 帧内分块过程的优化;3) 待编码系数的再处理。最后对 4 种优化算法进行单独、联合测试,给出优化算法的性能测试结果。

% 第 4 章开始介绍硬件实现,详细描述了所提出的优化算法在 H.265 硬件编码器上的实现框架和具体模块的实现方案,给出行为级仿真结果。同时介绍完整的 FPGA 原型验证方案,给出上板验证结果。

% 第 5 章对本课题的全部工作进行总结,并针对下一步的研究工作作出展望。

第 4 章开始介绍 H.265 硬件编码器的实现,详细描述了 H.265 硬件编码器的实现框架和部分关键模块的实现方案,并给出行为级仿真结果。

第 5 章介绍完整的 H.265 硬件编、解码系统的 FPGA 原型验证平台,详细描述平台验证方案,对验证结果进行分析。

第 6 章对本课题的全部工作进行总结,并针对下一步的研究工作作出展望。
