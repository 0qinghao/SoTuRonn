% !Mode:: "TeX:UTF-8"
\chapter{绪论}
\label{cha:c1}

\section{课题研究背景与意义}

\section{国内外研究现状}

\section{本课题的主要工作}
\cite{SAP-SAP1}
\section{本文章节安排}
本文由 6 个章节组成,具体安排如下:

本章为绪论,概述课题研究背景与意义,介绍本课题相关的国内外研究状况,并简要阐述本课题的工作。

第 2 章首先从整体上介绍视频编码技术的发展历史,其中主要介绍 H.265,亦会涉及到其前身 H.264 及近年仍在草拟标准的 H.266;同时从帧内/帧间预测、变换量化、熵编码、环路滤波和率-失真优化 5 个视频编码的核心手段展开,简单介绍技术要点;最后介绍部分学术界和工业界常用的软、硬件视频编解码相关的开源项目,本课题就是以其中部分开源项目为基础开展的。

第 3 章是本文的核心部分之一,该章详细地探讨了帧内无损编码优化的可行性。通过对帧内预测过程、帧内分块决策过程和系数编码过程的进一步分析,提出了 3 个优化方案,分别涉及 1) 帧内预测过程的优化;2) 帧内分块过程的优化;3) 待编码系数的再处理。最后对 3 种优化算法进行单独、联合测试,给出优化算法的性能。

第 4 章开始介绍硬件实现,详细描述了所提出的优化算法在开源 H.265 硬件编码器上的实现方案,并给出了行为级仿真结果。

第 5 章介绍完整的 H.265 硬件编、解码系统的 FPGA 原型验证平台,与算法的软件实行进行对比,对验证结果进行分析。

第 6 章对本课题的全部工作进行总结,并针对下一步的研究工作作出展望。
