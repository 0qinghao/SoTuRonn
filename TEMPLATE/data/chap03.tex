% !Mode:: "TeX:UTF-8"
\chapter{数学和定理环境}
\label{cha:theorem}
\section{数学宏包}
\LaTeX\ 最擅长处理的就是数学公式, \shuthesis\ 已经预加载了常用的数学宏包, 包括:
\begin{itemize}
    \item 美国数学学会系列宏包: \texttt{amsmath}, \texttt{amssymb}, \texttt{amsfonts}.
    \item 生成英文花体的宏包: \texttt{mathrsfs}.
    \item 数学公式中的黑斜体的宏包: \texttt{bm}.
    \item AMS 的补充宏包: \texttt{mathtools}.
\end{itemize}

\section{定理类环境}
给大家演示一下 \shuthesis\ 预定义的各种定理类环境.

\subsection{\shuthesis\ 预定义的定理类环境}
\begin{assumption}
    天地玄黄, 宇宙洪荒, 日月盈昃, 辰宿列张.
\end{assumption}

\begin{definition}
    寒来暑往, 秋收冬藏, 闰余成岁, 律吕调阳.
\end{definition}

\begin{proposition}
    云腾致雨, 露结为霜, 金生丽水, 玉出昆冈.
\end{proposition}

\begin{remark}
    天不言自高, 水不言自流.
\end{remark}

\begin{axiom}
    两点间直线段距离最短.
\end{axiom}

\begin{lemma}
    证明如下等式:
    \[
        \sum_{n=1}^{\infty}\frac{n-1}{\binom{2n}{n}}=\frac{1}{3}.
    \]
\end{lemma}

\begin{proof}
    注意到下面的恒等式:
    \[
        \frac{1}{\binom{2n}{n}}=(2n+1)\int_0^1[x(1-x)]^n\,dx,
    \]
    和
    \[
        \sum_{n=1}^{\infty}(2n+1)(n-1)y^n=\frac{(y-5)y^2}{(y-1)^3}.
    \]
    记 $y=x(1-x)$, 则
    \[
        \sum_{n=1}^{\infty}(2n+1)(n-1)x^n(1-x)^n=\frac{(x-x^2-5)(x-x^2)^2}{(x-x^2-1)^3}.
    \]
    所以有
    \begin{align*}
        \sum_{n=1}^{\infty}\frac{n-1}{\binom{2n}{n}} & =
        \int_0^1\left[\sum_{n=1}^{\infty}(2n+1)(n-1)x^n(1-x)^n\right]dx                                           \\
                                                     & =\int_0^1\frac{(x-x^2-5)(x-x^2)^2}{(x-x^2-1)^3}dx=\frac13.
    \end{align*}
\end{proof}

\begin{theorem}\label{the:theorem1}
    一元五次方程没有一般的代数解.
\end{theorem}

\begin{corollary}
    这是推论环境.
\end{corollary}

\begin{example}
    大家来看一个例子.
\end{example}

\begin{exercise}
    设 $a_i\geq0$, $b_i\geq0$, $i=1$, $2$, $\ldots$, $n$,
    且 $p>1$, $q>1$ 满足 $1/p+1/q=1$. 证明
    \[
        \sum_{i=1}^{n}a_ib_i\leq\left(\sum_{i=1}^{n}a_i^p\right)^{1/p}
        \cdot\left(\sum_{i=1}^{n}b_i^q\right)^{1/q},
    \]
    等号成立当且仅当 $a_i^p=cb_i^q$.
\end{exercise}

\begin{problem}
回答还是不回答, 是个问题.
\end{problem}